\chapter{Configuration objects definitions}
\label{appendix:A}
Parameters that can be defined in simulation configuration file are defined in \textit{configurationSchema.json} file. Simulation files have tree structure and consist of several root objects that have multiple child objects, but for simplicity and re-usability, in schema those parameters are defined as flat list.

\begin{lstlisting}[label=list:param-def,caption=Parameter definitions in schema, basicstyle=\footnotesize\ttfamily]
[
    "paramName1": {...},
    "paramName2": {...}
]
\end{lstlisting}

The order in which parameters are defined in schema does not matter, as long as all parameters that other parameters depend on are defined. There are several parameter types
\section{Simple parameter}
This parameter type consists of single floating, integer or string value. It cannot have any child parameters. Example of such parameters in simulation file - \textit{sigma}, \textit{gamma} and \textit{scale} are simple parameters.
\begin{lstlisting}[label=list:simple-param,caption=Simple parameter definition, basicstyle=\footnotesize\ttfamily]
"parameters": {
    "sigma": "5",
    "gamma": "2",
    "scale": "10"
}
\end{lstlisting}
Definition for such parameter in `configurationSchema.json` looks like this:
\begin{lstlisting}[label=list:simple-param-def,caption=Simple parameter definition, basicstyle=\footnotesize\ttfamily]
"sigma": {
    "type": "simple",
    "description": "Some parameter description",
    "parameterType": "int",
    "minValue": 0,
    "maxValue": 1000,
    "defaultValue": 1
},
\end{lstlisting}

\section{Enum parameter}
Enum parameter is parameter that changes its child parameters based on its class - \textit{saturation} is an example of such parameter - it has two possible classes \textit{`inverse\_generalized\_exponential} and \textit{genralized\_exponential}

\begin{lstlisting}[label=list:enum-param-ex1,caption=Enum parameter example, basicstyle=\footnotesize\ttfamily]
"saturation": {
    "class": "inverse_generalized_exponential",
    "parameters": {
        "sigma": "10",
        "gamma": "2",
        "scale": "1000"
    }
},
\end{lstlisting}

Same parameter with different class would look like this:
\begin{lstlisting}[label=list:enum-param-ex2,caption=Enum parameter example, basicstyle=\footnotesize\ttfamily]
"saturation": {
    "class": "generalized_exponential",
    "parameters": {
        "sigma": "5",
        "gamma": "2",
        "scale": "10"
    }
}
\end{lstlisting}
Definition for such parameter in schema:
\begin{lstlisting}[label=list:enum-param-def,caption=Enum parameter definition, basicstyle=\footnotesize\ttfamily]
"saturation": {
    "type": "enum",
    "description": "Some description",
    "possibleClasses": [
        "generalized_exponential",
        "inverse_generalized_exponential"
    ],
    "defaultValue": "generalized_exponential"
}
\end{lstlisting}
\section{Complex}
Complex - this parameter type has one or more simple, complex or enum children. In example below \textit{generalized\_exponential} is such parameter - it has 3 simple children.
\begin{lstlisting}[label=list:complex-param-ex,caption=Complex parameter example, basicstyle=\footnotesize\ttfamily]
"saturation": {
    "class": "generalized_exponential",
    "parameters": {
        "sigma": "5",
        "gamma": "2",
        "scale": "10"
    }
},
\end{lstlisting}
\textit{Mutator} is an example of complex parameter with complex children :
\begin{lstlisting}[label=list:complex-ex,caption=Complex parameter example, basicstyle=\footnotesize\ttfamily]
"mutator": {
    "efficiency": {
        "class": "uniform_step",
        "parameters": {
            "increase_length": "0.1",
            "decrease_length": "1.0"
        }
    },
    "resistance": {
        "class": "uniform_step",
        "parameters": {
            "increase_length": "0.1",
            "decrease_length": "1.0"
        }
    }
}
\end{lstlisting}
Definition of such parameter is analogous to enum parameter, but instead of \textit{possibleClasses} that parameter can have, array of \textit{childClasses} is defined:
\begin{lstlisting}[label=list:complex-def,caption=Complex parameter definition, basicstyle=\footnotesize\ttfamily]
"mutator": {
    "type": "complex",
    "description": "Mutator description",
    "childClasses": ["efficiency", "resistance"]
},
\end{lstlisting}