\chapter*{Streszczenie Pracy}\mbox{}\pdfbookmark[0]{Acronyms}{Acronyms.1}
Projekt \textit{SimBaD} (\textit{ang. Simulation Birth and Death - Symulacja Narodzin i Śmierci}) zajmuje się symulowaniem procesu proliferacji komórek rakowych. Proliferacja to inaczej zdolność rozmnażania się komórek, a niekontrolowana proliferacja może prowadzić do zmian nowotworowych.
Symulacja odbywa się w trzech etapach, każdy przeprowadzany przez oddzielny program. Pierwszy etap \textit{(SimBaD-CLI)} to program przeprowadza symulacje i generuj strumień danych, który jest następnie analizowany przez kolejny etap \textit{SimBaD-Analyzer}. Ostatni etap \textit{(SimBaD-Reports)} tworzy z przeanalizowanych danych raporty o przebiegu symulacji, w postaci wykresów. Proces symulacji jest sterowany plikiem konfiguracyjnym, i w zależności od wybranego kryterium stopu  (liczby komórek w systemie, albo czasu działania) może trwać bardzo długi czas. Chociaż przepływ danych w systemie nie jest skomplikowany, system jest złożony. Głównym źródłem złożoności jest ciężka instalacja, konfiguracja i obsługa systemu. Każdy program etapu ma swoje zależności w postaci innych programów czy bibliotek które użytkownik musi zainstalować by go uruchomić. Dodatkowo, przekazywanie danych pomiędzy etapami symulacji nie jest obecnie zautomatyzowane, i wymaga od użytkownika wykonania programów i skryptów symulacji z odpowiednimi argumentami. Z powodu złożoności systemu, jego obsługa wymaga od użytkownika znaczącej wiedzy technicznej. Instalacja i obsługa systemu dla użytkownika bez takiej wiedzy nie jest możliwa, bez znaczącej pomocy i wsparcia ze strony autorów. Celem pracy było stworzenie interfejsu, który ułatwiałby kontrolowanie i monitorowanie procesu Symulacji, oraz umożliwienie jego instalacji na różnych systemach operacyjnych.

Do umożliwienia instalacji na różnych systemach wykorzystano konteneryzację z wykorzystaniem platformy \textit{Docker}. Kontener taki można następnie uruchomić na każdym systemie operacyjnym, gdzie \textit{Docker} jest zainstalowany. 
W celu ułatwienia kontrolowania procesu symulacji stworzono webowy interfejs graficzny. Za pomocą tego interfejsu użytkownik może stworzyć prawidłowy plik konfiguracyjny i uruchomić proces symulacji. Po uruchomieniu symulacji, każdy kolejny etap rozpoczyna się automatycznie, a jego postęp może być monitorowany przez użytkownika. Po ukończeniu danego etapu, użytkownik może pobrać utworzone pliki wyjściowe, a w przypadku ostatniego etapu może dodatkowo wyświetlić wyniki symulacji w postaci wykresów. Dwa sposoby komunikacji serwera z programami przeprowadzającymi symulację zostały zaproponowane: umieszczenie programu  wykonywającego etap na tym samym kontenerze co serwer, oraz umieszczenie programu na oddzielnym kontenerze wraz z dodatkową warstwą sieciową, przez którą serwer po połączeniu \textit{HTTP} lub \textit{SSH} może zdalnie uruchomić etap. Pierwszy sposób wykorzystano do etapów \textit{SimBaD-CLI} i \textit{SimBaD-Reports}, drugi do \textit{SimBaD-Analyzer}. 

Proponowane rozszerzenie zaimplementowano, a jego działanie zostało przetestowane na systemach \textit{Windows} i \textit{GNU\textbackslash Linux}. Po zainstalowaniu i uruchomieniu systemu użytkownik jest w stanie przeprowadzić cały proces symulacji za pomocą kilku kliknięć.