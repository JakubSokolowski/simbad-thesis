\chapter{Conclusion}
\label{chapter:6}
\section{Encountered issues}
\subsection{Dead end of docker.host.internal}
\subsection{Growing size of containers}
\section{Future work}
\subsection{3D model of Cells}
From the simulation output, it is possible to generate 3D model of cancer cells in system. Currently, this model is rendered using CloudCompare. Final extension of the simulation pipeline would be to generate such model and render it in browser. The user could then rotate the cells, view its surface and cross-section. To manage such computationally-heavy processing, from client side a combination of WEB-GL and WASM could be used.
\subsection{WCSS integration}
Another important extension, is to integrate the proposed system with WCSS supercomputer. This would be possible through writing SSH Executors for each step, and from the SAS side, add switching between running the spark as subprocess, and submitting the job to internal WCSS queue.
\subsection{Better use of Celery}
Whole TaskExecutor concept can be made obsolete through better use of existing Celery Features. In Celery, multiple workers can connect to the same broker, and execute scheduled task. To achieve the same separation of concerns as in case of http extensions and task executors, multiple task queues can be defined. For example, separate task queue for CLI tasks and separate task queue for Analyzer tasks. Specific worker can be assigned to run only tasks from specific queue. In such configuration, celery manages the passing of the messages and results, instead of manually defining http extensions and endpoints.
\subsection{UI/UX}
There are many improvements and additions that can be made to the user interface. First one that comes to mind is a view that would allow to browse results of already completed simulations,as for now, only the latest result can be loaded. Another important addition would be a way to configure setup the system from the browser. The initial docker-compose up would start the client, and in client user, using some kind of configuration wizard would be able to setup the system. An array of small improvements can be made to existing UI components: possibility to search parameter in configuration editor, filter input for finding selected artifact, possibility to adjust the font size or fixing the flex-box in simulation view to fit on smaller screens and mobile devices. Another important addition, albeit requiring the modification of existing simulation components is displaying more accurate progress. The stop criterion for simulation can be number of cells in system, or elapsed time. The second option is trivial to implement, but the first one requires information about current cell number from SimBaD-CLI, and that currently is not given.
\subsection{CI}
Continous itnergration is very important for software development as it allows to make sure that system works in automated way. It also streamlines the release process, as each successfully master build could update hosted documentation and push new images to DockerHub. Currently some form of CI system is setup for two components, the SimBad-CLI and simbad-Client. Another step would be to add CI for other components, and add some form of integration tests for entire system. The tests would simulate user activity in the browser, but the simulation would be run on real components. 
