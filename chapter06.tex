\chapter{Conclusion}
\label{chapter:6}
\section{Encountered issues}
\subsection{Dead end of \textit{host.docker.internal}}
Significant effort was made to add additional component to system, that would handle container-host communication. This would allow for such features as opening the folder where the output files are located from browser, or starting and stopping containers from \textit{SimBaD-Client}. The first feature was implemented, however, it only worked when the \textit{SimBaD-Client} network mode was set to host in \textit{docker-compose}, and such solution is only available in \textit{Docker} for \textit{GNU\textbackslash Linux}. \textit{Docker} documentation proposes another solution, using the \textit{docker.host.internal} \textit{DNS} name that would resolve to host address. This solution, contrary to what official documentation may suggest, does not work. There exists an open issue on \textit{Docker} \textit{Github} repository concerning this problem, but it is not yet resolved \cite{DockfileIssue}. As the result, further development of this component was abandoned.
\subsection{Growing size of containers}
The biggest growth in container size happened while integrating \textit{SimBaD-Reports} step. The library for reading \textit{Parquet} files - \textit{Pyarrow}, could not be installed on the existing \textit{Alpine} with server and \textit{SimBaD-CLI}, due to dependency conflicts. The \textit{C++ Arrow} library that \textit{Pyarrow} depended on, had to be build with \textit{glibc}, which could not be installed on \textit{Alpine} as alpine uses different implementation of the Standard C Library - \textit{musl libc}. This lead to switch to \textit{Debian-Buster} as underlying container OS. The package manager for \textit{Python} was also changed from \textit{pip} to \textit{Conda}, to allow installation of prebuild binaries.
All those changes made only to allow reading data from \textit{Parquet} files doubled the size of container. However, it's worth noting that dependency issues were also encountered while developing the proposed system, and the system would not be able to work on the machine that it was developed on without \textit{Docker}.
\section{Future work}
\subsection{3D model of Cells}
From the simulation output, it is possible to generate 3D model of cancer cells in system. Currently, this model is rendered using \textit{CloudCompare}. Final extension of the simulation pipeline would be to generate such model and render it in browser. The user could then rotate the cells, view its surface and cross-section. To manage such computationally-heavy processing, from client side a combination of \textit{WEB-GL} and \textit{WebAssembly} could be used.
\subsection{WCSS integration}
Another important extension, is to integrate the proposed system with t\textit{WCSS} supercomputer. This would be possible through writing SSH Executors for each step, and from the \textit{SAS} side, add switching between running the spark as subprocess and submitting the job to internal \textit{WCSS} queue.
\subsection{Better use of Celery}
Whole \textit{TaskExecutor} concept can be made obsolete through better use of existing \textit{Celery} Features. In Celery, multiple workers can connect to the same broker, and execute scheduled task \cite{CeleryRouting}. To achieve the same separation of concerns as in case of http extensions and task executors, multiple task queues can be defined. For example, separate task queue for \textit{SimBaD-CLI} tasks and separate task queue for \textit{SimBaD-Analyzer} tasks. Specific worker can be assigned to run only tasks from specific queue. In such configuration, celery manages the passing of the messages and results, instead of manually defining \textit{HTTP} extensions and endpoints.
\subsection{UI/UX}
There are many improvements and additions that can be made to the user interface. First one that comes to mind is a view that would allow to browse results of already completed simulations,as for now, only the latest result can be loaded. Another important addition would be a way to configure setup the system from the browser. The initial docker-compose up would start the client, and in client user, using some kind of configuration wizard would be able to setup the system. An array of small improvements can be made to existing UI components: possibility to search parameter in configuration editor, filter input for finding selected artifact, possibility to adjust the font size or fixing the flex-box in simulation view to fit on smaller screens and mobile devices. Another important addition, albeit requiring the modification of existing simulation components is displaying more accurate progress. The stop criterion for simulation can be number of cells in system, or elapsed time. The second option is trivial to implement, but the first one requires information about current cell number from \textit{SimBaD-CLI}, and that currently is not given.
